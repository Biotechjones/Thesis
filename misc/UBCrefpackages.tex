%!TEX root = MJThesis.tex
%%%%%%%%%%%%%%%%%%%%%%%%%%%%%%%%%%%%%%%%%%%%%%%%%%%%%%%%%%%%%%%%%%%%%%
%% HYPERREF:
%% The hyperref package provides for embedding hyperlinks into your
%% document.  By default the table of contents, references, citations,
%% and footnotes are hyperlinked.
%%
%% Hyperref provides a very handy command for doing cross-references:
%% \autoref{}.  This is similar to \ref{} and \pageref{} except that
%% it automagically puts in the *type* of reference.  For example,
%% referencing a figure's label will put the text `Figure 3.4'.
%% And the text will be hyperlinked to the appropriate place in the
%% document.
%%
%% Generally hyperref should appear after most other packages

%% The following puts hyperlinks in very faint grey boxes.
%% The `pagebackref' causes the references in the bibliography to have
%% back-references to the citing page; `backref' puts the citing section
%% number.  See further below for other examples of using hyperref.
%% 2009/12/09: now use `linktocpage' (Jacek Kisynski): GPS now prefers
%%   that the ToC, LoF, LoT place the hyperlink on the page number,
%%   rather than the entry text.
%% The following is a directive for TeXShop to indicate the main file

\usepackage[hyperref=true,
            url=false,
            isbn=false,
            backref=true,
            firstinits=true,
            style=custom-numeric-comp,
            citereset=chapter,
            maxcitenames=3,
            maxbibnames=100,
            % refsection=section,
            backend=bibtex, % while checking on one of my (newest) systems, this option was needed to generate bibliography
            block=none]{biblatex}

    \usepackage{varioref}

    \usepackage[
		    bookmarksopen=true,
		    linktocpage=true,
		    urlcolor=linkcolor, % Color of URLs
		    citecolor=linkcolor, % Color of citations
		    linkcolor=linkcolor, % Color of links to other pages/figures
		    backref=page,
		    pdfpagelabels=true,
		    plainpages=false,
		    colorlinks=false, % Turn off all coloring by changing this to false
		    bookmarks=true,
		    pdfview=FitB]{hyperref}

		    %% The following change how the the back-references text is typeset in a
		    %% bibliography when `backref' or `pagebackref' are used
		    % \renewcommand\backrefpagesname{\(\rightarrow\) pages}
		    % \renewcommand\backref{\textcolor{greytext} \backrefpagesname\ }
    %%%%%%%%%%%%%%%%%%%%%%%%%%%%%%
    % Customisation
    %%%%%%%%%%%%%%%%%%%%%%%%%%%%%%
    % back reference text preceding the page number ("see p.")
    \DefineBibliographyStrings{english}{%
        backrefpage  = {see p.}, % for single page number
        backrefpages = {see pp.} % for multiple page numbers
    }

    % the followings activate 'custom-english-ordinal-sscript.lbx'
    % in order to print ordinal 'edition' suffixes as superscripts,
    % and adjusts (reduces) spacing between suffix and following "ed."
    \DeclareLanguageMapping{english}{custom-english-ordinal-sscript}
    \DeclareFieldFormat{edition}%
                       {\ifinteger{#1}%
                        {\mkbibordedition{#1}\addthinspace{}ed.}%
                        {#1\isdot}}

    % removes period at the very end of bibliographic record
    \renewcommand{\finentrypunct}{}

    % removes period after DOI and suppresses capitalization
    % of the word following DOI ("See p. xx" -> "see p. xx")
    \renewcommand{\newunitpunct}{\addspace\midsentence}

    \DeclareFieldFormat{journaltitle}{\mkbibemph{#1},} % italic journal title with comma
    \DeclareFieldFormat[inbook,thesis]{title}{\mkbibemph{#1}\addperiod} % italic title with period
    \DeclareFieldFormat[article]{title}{#1} % title of journal article is printed as normal text
    \DeclareFieldFormat[article]{volume}{\textbf{#1}\addcolon\space} % makes volume of journal bold and adds colon
    \DeclareFieldFormat{pages}{#1} % removes pagination (p./pp.) before page numbers
    %%%%%%%%%
% the command \upcite defined below prints footnote citation above punctuation
\newlength{\spc} % declare a variable to save spacing value
\newcommand{\upcite}[2][]{% new command with two arguments: optional (#1) and mandatory (#2)
        \settowidth{\spc}{#1}% set value of \spc variable to the width of #1 argument
        \addtolength{\spc}{-1.8\spc}% subtract from \spc about two (1.8) of its values making its magnitude negative
        #1% print the optional argument
        \hspace*{\spc}% print an additional negative spacing stored in \spc after #1
        \supershortnotecite{#2}}% print (cite) the mandatory argument
%%%%%%%%%
    % back reference text preceding the page number ("see p.")
    \DefineBibliographyStrings{english}{%
        backrefpage  = {see p.}, % for single page number
        backrefpages = {see pp.} % for multiple page numbers
    }
	%%%%%%%%%
	\usepackage{cleveref}
		\newcommand{\creflastconjunction}{, and\nobreakspace} %Oxford Comma for cref
    %%%%%%%%%
    % the followings activate 'custom-english-ordinal-sscript.lbx'
    % in order to print ordinal 'edition' suffixes as superscripts,
    % and adjusts (reduces) spacing between suffix and following "ed."
    \DeclareLanguageMapping{english}{custom-english-ordinal-sscript}
    \DeclareFieldFormat{edition}%
                       {\ifinteger{#1}%
                       {\mkbibordedition{#1}\addthinspace{}ed.}%
                       {#1\isdot}}    
