%% The following is a directive for TeXShop to indicate the main file
%!TEX root = ../MJThesis.tex
\acresetall

\chapter{OmpW of \textit{Caulobacter crescentus} functions as an outer membrane channel}
\label{ch:porin}
\begin{epigraph}
  \emph{``And so, progressively, the veil behind which Nature has so carefully concealed her secrets is being lifted where the carbohydrates are concerned.''} ---~H.\,Emil Fischer, 1902 Nobel prize lecture\\ My great-great-great-great-great-great-great-grand advisor.
\end{epigraph}
\section{Introduction} % (fold)
\label{sec:porin_introduction} 
    \lettrine[lines=2]{T}{he} cell-envelope of gram-negative bacteria consists of different layers. The inner or cytoplasmic membrane contains the respiration chain, proteins for the transport of nutrients and proteins involved in the synthesis of phospholipids, peptidoglycan and lipopolysaccharides (Beveridge, 1981; Nikaido and Vaara, 1985). The periplasmic space between the membranes is an aqueous compartment isoosmolar to the cytoplasm (Benz, 1994). It contains the peptidoglycan and a large number of different proteins. The outer membrane is composed of protein, lipid and \ac{LPS} (Beveridge, 1981). It typically contains only a few major proteins. Normally at least one of the constitutive outer membrane proteins is a porin, a general diffusion pore with a defined exclusion limit for hydrophilic solutes (Nikaido, 2003). In addition to constitutive porins, an outer membrane may contain porins that are induced under special growth conditions (Benz, 2001). They often form solute-specific channels and contain binding sites for neutral substrates such as carbohydrates (Benz et al., 1986; Ferenci et al., 1980), or nucleosides (Benz et al., 1986) and phosphate (Benz and Hancock, 1987; Hancock et al., 1987). Many of the specific porins are part of uptake and degradation systems, such as the maltose uptake system of Escherichia coli (Schwartz, 1987). 

    \ac{caulobacter}, a member of the alphaproteobacteria group, is a gram-negative bacterium found in oligotrophic aquatic environments (Curtis and Brun, 2010). \ac{caulobacter} has well been studied as a model of cell cycle and bacterial differentiation. (McAdams and Shapiro, 2003). Its genome sequence has been known for more than 10 years (Niermann et al., 2001). \ac{caulobacter} is an unusual gram-negative bacterium in that genes coding for typical general diffusion porins of the OmpF/C type of enteric gram-negative bacteria have not been identified in its genome (Lohmüller et al., 2008; Neugebauer et al., 2005; Niermann et al., 2001). Similarly, genes coding for specific porins such as Tsx or LamB are also absent. Instead, the genome of \ac{caulobacter} contains a large number of genes that code for TonB-dependent receptors (Eisenbeis et al., 2008; Neugebauer et al., 2005). More than 60 of these receptors have been identified (Lohmüller et al., 2008), which probably means that most of the nutrients from dilute environments are taken up actively by these systems. Examples for this are the uptake of maltose and N-acetylglucosamine into the cells (Eisenbeis et al., 2008; Lohmüller et al., 2008). 

    In this study we investigated whether the outer membrane of \ac{caulobacter} also contained a porin-like channel. The results suggested that despite the assumption that the \ac{caulobacter} outer membrane does not contain porins; a porin-like channel could be detected in artificial membranes with a single-channel conductance of about 125 \si{\pico\sievert} in 1 \si{\molar} \ce{KCl}. The protein responsible for channel formation was identified to be a member of the large OmpW family of outer membrane proteins. OmpW analogue proteins are found in many gram-negative bacteria. Two members of this family, OmpW of \ac{ecoli} and OprG of \ac{pseudomonas} have been crystallized and their 3D-structures are known at high resolution (3.5 and 2.7 and 2.4 \AA, respectively) (Albrecht et al., 2006; Hong et al., 2006; Touw et al., 2010). Here we show that OmpW of \ac{caulobacter} functions as a channel for cations, which is in sharp contrast to OprG of P. aeruginosa and OmpW of E. coli, which are believed to be plugged completely or involved in the transport of small, yet unknown hydrophobic molecules across the cell wall of these bacteria (Albrecht et al., 2006; Hong et al., 2006; Touw et al., 2010). The 3D-structure of OmpW of \ac{caulobacter} was modeled on the basis of the known structures of OmpW of \ac{ecoli} and OprG of \ac{pseudomonas} (Hong et al., 2006; Touw et al., 2010). The results indicated that OmpW of \ac{caulobacter} could have a larger diameter and more hydrophilic interior than the two crystallized members of the OmpW family.

\section{Methods and Materials}
\label{sec:porin_methods}
\subsection{Growth and maintenance of microorganisms} 
\label{sub:porin_growth}
Caulobacter crescentus CB15 NA1000 353$\Phi$ (JS1013) carries an amber mutation in the gene rsaA resulting in S-layer deficiency (Ford et al., 2007). The strain was grown to mid log phase (\ac{OD600} = 0.8) in \ac{PYE} (Poindexter, 1964) at 30\si{\degreeCelsius} in 2.8 \si{\litre} Fernbach flasks containing 1250 \millilitre medium, shaken at 100 rpm.

\subsection{Outer membrane enriched preparations}
\label{sub:porin_omp_prep}
Cells were pelleted by centrifugation at 12,400 x g for 10 min. Cell pellets were washed by suspension with distilled water and repelleted. The pellets were resuspended in 1/10 original culture volume of \ac{PBS} (Maniatis et al., 1982) amended with 10 \millimolar \ac{EDTA}, agitated at room temperature for 5 min and then centrifuged at 15,300 x g for 15 min. The supernatant was retrieved and re-centrifuged to clarify. The supernatant was then ultracentrifuged at 184,000 x g for 2 h. Glassy pellets formed which were suspended in 1/100 original culture volume in 10 \millimolar Tris pH 8.0. This treatment led preferentially to the disruption of the outer membrane and periplasmic contents were released without significantly releasing cytoplasmic contents.

\subsection{Crude membrane preparations}
\label{sub:porin_crude_preps}
For comparison to the \ac{PBS}-\ac{EDTA} membrane enrichment method, crude membrane preparations were prepared from 5 \millilitre of mid logarithmic culture. The culture was sonicated at 50\% intensity for 5 x 30 sec bursts. DNAse and RNAse were added to final concentrations of 0.06 \mgperml and 0.60 \mgperml respectively, and incubated at 37\si{\degreeCelsius} for 1 h. The preparation was then ultracentrifuged for 2 h at 107,000 x g. A glassy pellet formed which was resuspended in 200 \microlitre of distilled water.

\subsection{Isolation and purification of the channel-forming protein from enriched outer membranes}
\label{sub:porin_isolation}
The enriched outer membranes prepared by \ac{PBS}-\ac{EDTA} extraction was inspected for channel-forming activity by treatment with the detergent \ac{LDAO}. The detergent extracts of the enriched OM showed rapid channel formation in the lipid bilayer assay. The protein responsible for channel formation was identified by preparative \ac{SDS-PAGE}. Highest channel-forming activity was observed in the molecular mass range between 20 and 25 kDa.
\ac{SDS-PAGE}. 
Analytical and preparative \ac{SDS-PAGE} was performed according to (Laemmli, 1970). The gels were stained with Coomassie brilliant blue or Colloidal Coomassie blue (xx).

\subsection{Tryptic digestion and peptide sequencing}
\label{sub:porin_tryptic}
The pure 22 kDa protein eluted from preparative \ac{SDS-PAGE} was subjected to amino acid sequence analysis. Direct sequencing was not possible presumably because of blocking of the N-terminus. The 22 kDa protein was then cleaved with trypsin as described (Eckerskorn and Lottspeich, 1989). The peptides were separated by reversed phase HPLC on a Purospher RP18 encapped 5 \si{\micro\metre} column (Merck, Darmstadt, Germany) using a solvent gradient from 0 to 60\% acetonitrile in 0.1\% trifluoroacetic acid/water (v/v). The flow rate was 60 \si{\micro\litre\per\minute} and UV-detection was performed at 206 \si{\nano\metre}. The amino acid sequence analysis of the tryptic peptides was performed using an ABI 472A protein sequencer (Applied Biosystems, Langen, Germany).

\subsection{Lipid bilayer experiments}
\label{sub:porin_bilayer}
The method used for the reconstitution experiments using black lipid bilayer membranes has been described previously (Benz et al., 1978). The membranes were formed from a 1\% (w/v) solution of diphytanoyl-\ac{PC} (Avanti Polar Lipids, Alabaster, AL, U.S.A.) in n-decane. The membrane current was measured with a pair of calomel electrodes switched in series with a voltage source and an electrometer (Keithley 617). For single-channel recordings the electrometer was replaced by a highly sensitive current amplifier (Keithley 427). Zero-current membrane potentials were measured by establishing a salt gradient across membranes containing 100 to 1000 channels, as has been described earlier (Benz et al., 1979; 1985) using a high impedance electrometer (Keithley 617).

\subsection{Modeling of the OmpW structure}
\label{sub:porin_modeling}
The possible 3D-structure of OmpW of \ac{caulobacter} was derived using the homology modeling approach. Three dimensional model of \ac{caulobacter} OmpW was built using \texttt{Modeller} program (Eswar, et al., 2006) taking \ac{ecoli} OmpW as a template structure (Hong et al., 2006).
 
