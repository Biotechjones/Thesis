
%% The following is a directive for TeXShop to indicate the main file
%!TEX root = ../MJThesis.tex
\acresetall
\resetlinenumber[1]
\chapter{Conclusions}
\label{ch:conclusions}
\begin{epigraph}
  \emph{``We must never let ourselves fall in thinking `ignorabimus' (`We shall never know'), but must have every confidence that the day will dawn when even those processes of life which are still a puzzle today will cease to be inaccessible to us natural scientists.''}\\ ---~Eduard~Buchner,~1907~Nobel~lecture 
\end{epigraph}
\section{Relevance and contribution to the field}\label{sec:relev-contr-field} 

The \textit{Caulobacter} field is dominated by research groups teasing apart the fine details of the \caulobacter{} lifecycle and the panoply of regulatory networks that control every aspect of bacterial biology. The \ac{S-layer}
 and the outer membrane often are neglected by the rest of the field. The focus on laboratory cultures, and not the bacteria in the environment, is probably the reason for this. A pure culture living in rich medium does not have as complex of a relationship with its environment as a wild specimen competing in a hostile setting. Its the cell envelope that acts as the barrier between the world and the cell. Looking at the \caulobacter cell envelope in the wild was well beyond the scope of this endeavor, we studied the cell envelope because it is an incredible structure in its own right, even in the lab. The cell envelope of \caulobacter{} was particularly suitable for inspection because it had a \acl{S-layer}, various \aclp{PS} including an \ac{OPS} of unknown structure, and apparently no porins. There is the added aspect that \caulobacter{}, and its \ac{S-layer}, are actively being pursued for biotechnology applications. This means that any new findings could quickly be applied towards exciting biological products. This work presented in chapters \ref{ch:crystal} \ref{ch:lps}, and \ref{ch:porin} present our efforts to improve the structure and composition knowledge of the \ac{S-layer}, \ac{LPS}, and outer membrane proteins in \caulobacter{}.
 
% The work in this dissertation also is focused on pure cultures in a laboratory setting, to its detriment. \Cref{ch:crystal} would have benefitted from information about how \acp{S-layer} increase fitness for wild \caulobacter{}.  \cref{ch:lps}  
\Cref{ch:crystal} lays out the effort to solve the structure of RsaA. Protein crystallography is the only approach applicable to tackle the \caulobacter{} \ac{S-layer} protein. Unluckily, \ac{S-layer} proteins are notoriously resistant to crystallography. The work of Pavkov \etal{}\upcite[]{Pavkov20081226} and Baranova \etal{}\upcite[]{baranova2012sbsb} demonstrated that \ac{S-layer} proteins are not impossible to crystallize as long as the ability for the proteins to form \acp{S-layer} is blocked. Our strategy to tackle \ac{S-layer} formation was a truncation approach, removing the N-terminal 222 \ac{aa}, which mirrored the successful approach used to solve SbsC\upcite[.]{Pavkov20081226} Novel protein expression and purification techniques had to be developed to produce well-behaved protein suitable for crystallography. Once we started to grow crystals, RsaA was set to be the third bacterial \ac{S-layer} protein to be structurally solved, the second with the symmetry determining domain intact, the first from a Gram-negative, and the largest. Lamentably, finding a phasing solution became the issue that was insurmountable. No phasing approach was successful, not the halide soaks that we had predicted would work, nor the lanthanide soaks that had worked for other groups\upcite[,]{baranova2012sbsb} nor any other compound. The C-terminal 804 \ac{aa} of RsaA have been crystallized, but the structure has not. Solving this structure would provide the field with unique window into a self-assembling protein lattice from a commonly studied Gram-negative model organism. 

\ac{LPS} structures are not as rare as \ac{S-layer} structures and yet the complete structure of the \caulobacter{} \ac{LPS} had never been determined when we started the work covered in \cref{ch:lps}. The \caulobacter{} lipid A structure was solved by Smit \etal{} in 2008\upcite[,]{caulobacterlipida} revealing a novel structure lacking phosphates and centered on a di-diaminoglucose backbone. The structures of the \ac{OS} and \ac{OPS} were identified for analysis because their fine structures had never been determined and because they comprise the anchor for the \caulobacter{} \ac{S-layer}\upcite[.]{walker94} An initial component analysis of the core \ac{OS} was undergone in 1992\upcite[.]{ravenscroftlps} The \ac{OPS} had never been studied directly but it was predicted that it was composed of perosamine\upcite[.]{awramgenes} Through a very successful collaboration with Dr. Evgeny Vinogradov the structures of the \caulobacter{} \ac{OS} and \ac{OPS} are now determined. The core \ac{OS} has a three armed configuration (see \cref{fig:lpscore} on \cpageref{fig:lpscore}). The final determined components of the core \ac{OS} are different than were first reported\upcite[,]{ravenscroftlps} especially in the finding that there is no phosphate in the core \ac{OS}. The first arm is a disaccharide of glucuronic acid and galactose. The two negative charges in the galacturonic acids in the lipid A, the one in the central Kdo, and the one in the glucuronic acid are the only charges present in the \caulobacter{} \ac{LPS}. The second arm is a trisaccharide of DD-heptose, mannose, and DL-heptose. The third arm is a single DL-heptose, linked to the C-7 position of the Kdo moiety. This triple substitution on the Kdo appears to be unique in the literature. Substitutions at all same positions (C-4, C-5, and C-7) in Kdo but not all three at once. The \ac{OPS} has a repeating heptasaccharide structure (see \cref{fig:lpsops} on \cpageref{fig:lpsops}). Seven sugar \ac{OPS} subunits are amongst the largest sizes that subunits are known to exist in. For example, there is just one known example of a seven sugar \ac{OPS} in \ecoli{}, and none larger\upcite[.]{stenutz2006structures} The \caulobacter{} \ac{OPS} terminates in one of either two end groups (see \cref{fig:lpsends} on \cpageref{fig:lpsends}). Both the \ac{OPS} and end groups contain an abundance of hydrophobic sugars, \ie N-acetylperosamine, rhamnose, methylglucose, dimethyl-N-acetylperosamine. There may be a role for the hydrophobicity to play in supporting an \ac{S-layer}. The \ac{OPS} from \textit{Aeromonas hydrophila} AH-1\upcite[,]{merino2015molecular} another alphaproteobacterium with an \ac{S-layer}, supports this hypothesis; one in two monosaccharides in its \ac{OPS} is the deoxysugar acetlyrhamnose. In the process of studying the \caulobacter{} \ac{LPS} we discovered a previously unknown rhamnose based polysaccharide (see \cref{fig:lpsrhamnan} on \cpageref{fig:lpsrhamnan}). Just the presence of the polysaccharide alone is of interest especially in light of recent interest in some of the \acp{EPS} from \caulobacter{}\upcite[.]{ardissone2014cell}  

\Cref{ch:porin} reports our characterization of OmpW, a porin in the outer membrane of \caulobacter{}. Previous to this work, no porin had been reported in \caulobacter{}. The wisdom on the subject was that \caulobacter{} primarily, if not solely, used active transport systems to take up nutrients from the environment\upcite[.]{lohmiller2008tonb} We found that OmpW acts as a passive protein channel in the outer membrane (see \cref{fig:porin-20ksinglechannel} on \cpageref{fig:porin-20ksinglechannel}). This is particularly interesting because the homologous proteins in \ecoli{} and \ac{pseudomonas} do not have any pore forming activity\upcite[]{hong2006outer, touw2010crystal} and no clear function at all.  The activity we observed is possibly due to a wider pore and specifically a lack of a plug seen in the other bacteria.  

\section{Future directions}\label{sec:future-directions}

