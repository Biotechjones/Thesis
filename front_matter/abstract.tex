%% The following is a directive for TeXShop to indicate the main file
%%!TEX root = diss.tex

\chapter{Abstract}

Classically, outer membranes are half lipid, half protein, and the outmost
layers of Gram-negative bacteria. For \textit{Caulobacter crescentus} the outer
membrane is the penultimate layer beneath a protein surface layer (S-layer). The
S-layer of the caulobacter cell envelope is an exciting platform for high
density peptide display and biotechnology development. We focused on elucidating
the structure of the outer membrane by crystallizing the S-layer protein, RsaA; solving the structure of the the lipopolysaccharide; and characterizing a newly discovered porin, OmpW. 

S-layer proteins are highly resistant to crystallization, because wo-dimensional
S-layer formation out competes three-dimensional crystal formation. To achieve a
crystallisable form of RsaA, a C-terminal truncation version was constructed and
expressed in the native host, \textit{C. crescentus}. The secreted protein was
prone to aggregation, so low agitation and slow concentration protocols had to
be developed. The RsaA truncate produced large crystals that diffracted to <2.5
\AA. Solving the phases proved to be a serious hurdle and the final protein structure remains unsolved.

The lipopolysaccharide of \textit{C. crescentus} is the anchor that supports the
S-layer. The structure of the lipid A portion was solved previously but
structures for the core oligosaccharide and the O-polysaccharide had not been deduced. In collaboration with Dr. Evgeny
Vinogradov, these remaining structures were solved.
The core oligosaccharide has a branched heptasaccharide structure. The O-polysaccharide is a heptasaccharide containing the dideoxy sugar N-acetylperosamine. Additionally, a rhamnan polysaccharide was discovered and its structure was determined. 

Porins, non-specific passive protein channels, are significant components of classical Gram-negative outer membranes. Despite this, no porin had ever been identified in \textit{C. crescentus}. We report the identification and characterization of the porin OmpW in \textit{C. crescentus}.  OmpW has low conductance of 125 pSv in 1 M KCl, that is interesting because homologous porins in other bacteria have no detectable pore-forming activity.

The cell envelopes of bacteria are remarkable structures; the work here illuminates the unique structures present in the caulobacter envelope.
