%% The following is a directive for TeXShop to indicate the main file
%%!TEX root = diss.tex

\chapter{Abstract}

Classically, outer membranes are half lipid, half protein, and the outmost layers of Gram-negative bacteria. For \textit{Caulobacter crescentus} the outer membrane is the penultimate layer beneath protein surface layer (S-layer) composed of the protein RsaA. The caulobacter cell envelope's S-layer is an exciting platform for high density peptide display and biotechnology development. We focused on elucidating the structure of the cell envelope by crystallizing the S-layer protein, RsaA; solving the structure of the the lipopolysaccharide; and functionally characterizing a newly discovered porin, OmpW. 

Bacterial S-layer proteins are highly resistant to crystallization; only two successful examples exist in the literature. Two-dimensional S-layer formation out competes three-dimensional crystal formation. To achieve a crystallisable form of RsaA, a C-terminal truncation version was constructed and expressed in the native host, \textit{C. crescentus}. The secreted protein was prone to aggregation, so low agitation and slow concentration protocols had to be developed. The RsaA truncate produced large crystals that diffracted to <2.5 \AA. Solving the phases proved a serious hurdle. Multiple phaisng strategies were pursued, but the final protein structure remains unsolved.

The lipopolysaccharide of \textit{C. crescentus} is the anchor that supports the S-layer. The structure of the lipid A molecule was solved previously but the structures of the core oligosaccharide and O-polysaccharide had never been deduced. In collaboration with Dr. Evgeny Vinogradov, the remaining unsolved structures in the lipopolysaccharide were solved. The core oligosaccharide has a heptasaccharide structure centering on a rare triple-substituted Kdo residue. The O-polysaccharide is a heptasaccharide containing the dideoxy sugar N-acetylperosamine. Additionally, a previously unknown rhamnan polysaccharide was discovered and its structure was determined. 

Porins, non-specific passive protein channels, are significant components of classical Gram-negative outer membranes. Despite this, no porin had ever been identified in the \textit{C. crescentus}. Here we report the identification and characterization of the porin OmpW in \textit{C. crescentus}.  OmpW has low conductance of 125 pSv in 1 M KCl, this is interesting because the homologous porins in other bacteria have no detectable pore-forming activity.

The cell envelopes of bacteria are remarkable structures; the work in thesis presents new information about the unique structures in the caulobacter envelope. 
