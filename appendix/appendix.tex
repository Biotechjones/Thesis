\chapter*{Appendix}
\renewcommand{\thechapter}

% \section{The amino aid sequence of RsaA from \caulobacter}

\addcontentsline{toc}{chapter}{Appendix}
\setcounter{figure}{0}
\renewcommand{\thefigure}{A.\arabic{figure}}
\begin{figure}[htb]
  	\begin{center}
\label{app:rsaseq}
\texttt{\singlespacing\small\underline{MAYTTAQLVTAYTNANLGKAPDAATTLTLDAYATQTQTGGLSDAAALTNTLKLVNSTTAV}\hfill-60~~\\
\underline{AIQTYQFFTGVAPSAAGLDFLVDSTTNTNDLNDAYYSKFAQENRFINFSINLATGAGAGA}\hfill-120~\\
\underline{TAFAAAYTGVSYAQTVATAYDKIIGNAVATAAGVDVAAAVAFLSRQANIDYLTAFVRANT}\hfill-180~\\
\underline{PFTAAADIDLAVKAALIGTILNAATVSGIGGYATATAAMI}NDLSDGALSTDNAAGVNLFT\hfill-240~\\
AYPSSGVSGSTLSLTTGTDTLTGTANNDTFVAGEVAGAATLTVGDTLS\textbf{GGAGTDVLN}WVQ\hfill-300~\\
AAAVTALPTGVTISGIETMNVTSGAAITLNTSSGVTGLTALNTNTSGAAQTVTAGAGQNL\hfill-360~\\
TATTAAQAANNVAVDGGANVTVASTGVTSGTTTVGANSAASGTVSVSVANSSTTTTGAIA\hfill-420~\\
VTGGTAVTVAQTAGNAVNTTLTQADVTVTGNSSTTAVTVTQTAAATAGATVAGRVNGAVT\hfill-480~\\
ITDSAAASATTAGKIATVTLGSFGAATIDSSALTTVNLSGTGTSLGIGRGALTATPTANT\hfill-540~\\
LTLNVNGLTTTGAITDSEAAADDGFTTINIAGSTASSTIASLVAADATTLNISGDARVTI\hfill-600~\\
TSHTAAALTGITVTNSVGATLGAELATGLVFT\textbf{GGAGADSILL}GATTKAIVMGAGDDTVTV\hfill-660~\\
SSATLGAGGSVN\textbf{GGDGTDVLV}ANVNGSSFSADPAFGGFETLRVAGAAAQGSHNANGFTAL\hfill-720~\\
QLGATAGATTFTNVAVNVGLTVLAAPTGTTTVTLANATGTSDVFNLTLSSSAALAAGTVA\hfill-780~\\
LAGVETVNIAATDTNTTAHVDTLTLQATSAKSIVVTGNAGLNLTNTGNTAVTSFDASAVT\hfill-840~\\
GTGSAVTFVSANTTVGEVVTIR\textbf{GGAGADSLT}GSATANDTII\textbf{GGAGADTLV}YTGGTDTFT\textbf{G}\hfill-900~\\
\textbf{GTGADIFD}INAIGTSTAFVTITDAAVGDKLDLVGISTNGAIADGAFGAAVTLGAAATLAQ\hfill-960~\\
YLDAAAAGDGSGTSVAKWFQFGGDTYVVVDSSAGATFVSGADAVIKLTGLVTLTTSAFAT\hfill-1020\\
EVLTLA\hfill-1026\\
  }
   	\end{center}
   	\caption[RsaA, amino acid sequence]{
   The amino acid sequence of the \ac{S-layer} protein, RsaA from \caulobacter{} NA1000. GeneID:~\texttt{CCNA\_01059}. Ensembl Accession number: \texttt{ACL94524}. The encoding gene has the coordinates of 1 159 693 bp--1 162 773 bp on the forward strand of the \caulobacter chromosome. The underlined sequence corresponds to the amino acids 1--222, the section of the protein that was removed to generate a crystallizable C-terminal fragment. The bolded sequences are the \ac{rtx} motifs in RsaA. For the crystallization of RsaA, see \cref{ch:crystal} \cpageref{ch:crystal}.}
   	
\end{figure}   
% \section{The amino acid sequence of OmpW from \caulobacter}
\begin{figure}[htb]
  	\begin{center}
\label{app:ompwseq}
\texttt{\singlespacing\small  MKKLALSLVAFGALAAGAAQAQDFTPNAKGDLIVHARLTQVAPAKDAAILTAAGANSGLK\hfill-60~\\
AHVGNDIKPTLGFTYFLTDKVAVEAILGTTEHNIRAQGPGTDVLVHKTWVLPPVVTLQYH\hfill-120\\
PLPASQVSPYVGAGLNYMLFYSGKNKNGFTVKVDDGVGYALQAGVNIKMKNSWLVNADV\textbf{K}\hfill-180\\
KVYFSTDAKINGGALKAKVDLDPVVASIGLSRKF\hfill-214\\
  }
   	\end{center}
   	\caption[OmpW, amino acid sequence]{
   The amino acid sequence of the outer membrane channel, OmpW from \caulobacter NA1000. GeneID: \texttt{CCNA\_01475}. Ensembl Accession number: \texttt{ACL94940}. The encoding gene has the coordinates of 1 582 599 bp--1 583 243 bp on the forward strand of the \caulobacter chromosome. The bolded `K' is the location of the putative hydrophobic gate in \ecoli and \ac{pseudomonas}, where that residue encodes for a tryptophan; in \caulobacter that residue is a lysine. For our investigations into OmpW, see \cref{ch:porin} on \cpageref{ch:porin}}
\end{figure}   
