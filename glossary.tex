%% The following is a directive for TeXShop to indicate the main file
%%!TEX root = diss.tex

\DeclareAcronym{LPS}{long = lipopolysaccharide, 
short = lps,
short-format = \scshape}
\DeclareAcronym{PS}{long = polysaccharide, 
short = ps,
short-format = \scshape}
\DeclareAcronym{OPS}{long = O-specific polysaccharide, 
short = ops,
short-format = \scshape}
\DeclareAcronym{OS}{long = oligosaccharide, 
short = os,
short-format = \scshape}
\DeclareAcronym{UV}{long = ultraviolet Light, 
short = uv,
short-format = \scshape}
\DeclareAcronym{MALDI-TOF}{long = matrix assisted laser desorption/ionization-time of flight mass spectroscopy, 
short = maldi-tof,
short-format = \scshape}
\DeclareAcronym{S-layer}{long = protein surface layer, 
short = S-layer}
\DeclareAcronym{GC-MS}{long = gas chromatography-mass spectroscopy, 
short = gc-ms,
short-format = \scshape}
\DeclareAcronym{NMR}{long = nuclear magnetic resonance spectroscopy, 
short = nmr,
short-format = \scshape}
\DeclareAcronym{SDS-PAGE}{long = sodium dodecyl sulfate-polyacrylamide gel electrophoresis, 
short = sds-page,
short-format = \scshape}
\DeclareAcronym{PBS}{long = phosphate-buffered saline, 
short = pbs,
short-format = \scshape}
\DeclareAcronym{EDTA}{long = ethylenediaminetetraacetic acid, 
short = edta,
short-format = \scshape}
\DeclareAcronym{COSY}{long = correlation spectroscopy, 
short = cosy,
short-format = \scshape}
\DeclareAcronym{gCOSY}{long = gradient correlation spectroscopy, 
short = g\textsc{cosy}}
\DeclareAcronym{TOCSY}{long = total correlation spectroscopy, 
short = tcosy,
short-format = \scshape}
\DeclareAcronym{ROESY}{long = rotating frame nuclear Overhauser effect spectroscopy, 
short = roesy,
short-format = \scshape}
\DeclareAcronym{HSQC}{long = heteronuclear single quantum coherence, 
short = hsqc,
short-format = \scshape}
\DeclareAcronym{gHSQC}{long = gradient heteronuclear single quantum coherence, 
short = g\textsc{hsqc}}
\DeclareAcronym{HMBC}{long = heteronuclear multiple bond coherence, 
short = hmbc,
short-format = \scshape}
\DeclareAcronym{gHMBC}{long = gradient heteronuclear multiple bond coherence, 
short = g\textsc{hmbc}}
\DeclareAcronym{NOE}{long = nuclear Overhauser effect, 
short = noe,
short-format = \scshape}
\DeclareAcronym{NOESY}{long = nuclear Overhauser effect spectroscopy, 
short = noesy,
short-format = \scshape}
\DeclareAcronym{HMQC}{long = heteronuclear multiple-quantum correlation spectroscopy, 
short = hmqc,
short-format = \scshape}
\DeclareAcronym{TLC}{long = thin-layer chromatography, 
short = tlc,
short-format = \scshape}
\DeclareAcronym{ABC}{long = ATP-binding cassette, 
short = abc,
short-format = \scshape}
\DeclareAcronym{EPS}{long = extracellular polysaccharide, 
short = eps,
short-format = \scshape}
\DeclareAcronym{ESI}{long = electrospray ionization, 
short = esi,
short-format = \scshape}
\DeclareAcronym{TFA}{long = trifluoroacetic acid, 
short = tfa,
short-format = \scshape}
 
\DeclareAcronym{man}{
    short = Man ,
    long = mannose , 
class = sugar
}
\DeclareAcronym{rha}{
    short = Rha ,
    long = rhamnose , 
    class = sugar
}
\DeclareAcronym{per}{
    short = Per ,
    long = perosamine , 
    class = sugar
}
\DeclareAcronym{glc}{
    short = Glc ,
    long = glucose , 
    class = sugar
}
\DeclareAcronym{glca}{
    short = GlcA ,
    long = glucuronic acid , 
    class = sugar
}
\DeclareAcronym{gala}{
    short = GalA ,
    long = galacturonic acid , 
    class = sugar
}
\DeclareAcronym{hep}{
    short = Hep ,
    long = heptose , 
    class = sugar
}
\DeclareAcronym{glcome}{
    short = Glc3Me ,
    long = 3-O-methylglucose , 
    class = sugar
}
\DeclareAcronym{MS}{
    short = MS ,
    long = mass spectroscopy, 
    short-format = \scschape 
} 

% 
% \newacro{ANOVA}[ANOVA]{Analysis of Variance\acroextra{, a set of
%   statistical techniques to identify sources of variability between groups.}}
% \newacro{API}[API]{application programming interface}
% \newacro{GOMS}[GOMS]{Goals, Operators, Methods, and Selection\acroextra{,
%   a framework for usability analysis.}}
% \newacro{TLX}[TLX]{Task Load Index\acroextra{, an instrument for gauging
%   the subjective mental workload experienced by a human in performing
%   a task.}}
% \newacro{UI}[UI]{user interface}
% \newacro{UML}[UML]{Unified Modelling Language}
% \newacro{W3C}[W3C]{World Wide Web Consortium}
% \newacro{XML}[XML]{Extensible Markup Language}
